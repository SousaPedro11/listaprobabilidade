\pagestyle{cap1}
\pagenumbering {arabic}

\section*{Variável Aleatória Discreta}
\addcontentsline{toc}{section}{Variável Aleatória Discreta}

\begin{enumerate}
\item Em um determinado condomínio residencial $30\%$ das famílias não tem filhos, $40\%$ tem $1$ filho, $20\%$ tem $2$ filhos e $10\%$ tem $3$ filhos. Seja $X$ o número de filhos de uma família sorteada ao acaso dentro desse condomínio residencial.

\solv{30\% não tem filhos\\40\% tem 1 filho\\20\% tem 2 filhos\\10\% tem 3 filhos}

    \begin{enumerate}[label=\alph*)]
		\item Determine a função de probabilidade e a distribuição acumulada de $X$.
		
		\solv{x: nº de filhos\\
		\{0,1,2,3\} $\, \to\,$ possíveis valores que x pode assumir.\\$p(0) = 0,3;\ p(1) = 0,4;\ p(2)=0,2;\ p(3)=0,1$
		
		\begin{center}
			\begin{tabular}{|c|c|c|c|c|}
            	 		\hline
           	 		$X$            & 0    & 1    & 2     & 3    \\ \hline
            	 		$P(x_{i})$ & 0,3 & 0,4 & 0,2 & 0,1 \\ \hline
        		 	\end{tabular}
    		\end{center}
		}
		\obs{Falta o gráfico}
		
		\item Calcule a esperança e o desvio padrão de $X$.
		
		\solv{
			\begin{equation*}
				{E(X) = \sum^{n}_{i=1} x_{i}p(x_{i})};\ Var(x) = \sum^{n}_{i=1} x_{i}^{2}p(x_{i}) - [E(X)]^{2};\ DP = \sqrt{Var(X)}
			\end{equation*}
			
			\begin{center}
				\begin{tabular}{cc|c|}
					\hline
					\multicolumn{1}{|c|}{$X$}	     & $p(x_{i})$	              	      & $x_{i}.p(x_{i})$                   \\ \hline
					\multicolumn{1}{|c|}{0}         & 0,3                          	      & 0                        \\ \hline
					\multicolumn{1}{|c|}{1}         & 0,4                     	      & 0,4                      \\ \hline
					\multicolumn{1}{|c|}{2}         & 0,2                      	      & 0,4                      \\ \hline
					\multicolumn{1}{|c|}{3}         & \multicolumn{1}{c|}{0,1} & \multicolumn{1}{c|}{0,3} \\ \hline
					\multicolumn{1}{|c}{Total}    & \multicolumn{1}{c|}{}       & \multicolumn{1}{c|}{1,1}    \\ \hline
				\end{tabular}
				\hfil
				\begin{tabular}{cc|c|}
					\hline
					\multicolumn{1}{|c|}{$X$}	     & $p(x_{i})$	              	      & $x_{i}^{2}.p(x_{i})$                   \\ \hline
					\multicolumn{1}{|c|}{0}         & 0,3                          	      & 0                        \\ \hline
					\multicolumn{1}{|c|}{1}         & 0,4                     	      & 0,4                      \\ \hline
					\multicolumn{1}{|c|}{2}         & 0,2                      	      & 0,8                      \\ \hline
					\multicolumn{1}{|c|}{3}         & \multicolumn{1}{c|}{0,1} & \multicolumn{1}{c|}{0,9} \\ \hline
					\multicolumn{1}{|c}{Total}    & \multicolumn{1}{c|}{}       & \multicolumn{1}{c|}{2,1}    \\ \hline
				\end{tabular}
			\end{center}

			${\mathbfit{E(X) = 1,1}};\\ Var(X) = 2,1 - (1,1)^{2}\ $$\, \to\,$$\ 2,1 - 1,21 = 0,89$	\\
			${\mathbfit{DP = \sqrt{0,89} \, \to\, 0,9434}}$	
		}
		
	\end{enumerate}

	\solv{}

\item Um indivíduo que possui um seguro de automóvel de uma determinada empresa é selecionado aleatoriamente. Seja $Y$ o número de infrações ao código de trânsito para as quais o indivíduo foi reincidente nos últimos $3$ anos. A função de de probabilidade de $Y$ é:

    \begin{center}
        \begin{tabular}{|c|c|c|c|c|}
            \hline
            $Y$      & 0    & 1    & 2    & 3    \\ \hline
            $P(Y=y)$ & 0,60 & 0,25 & 0,10 & 0,05 \\ \hline
        \end{tabular}
    \end{center}
    
    \begin{enumerate}[label=\alph*)]
    		\item Calcule $E(Y)$.
    		
			\solv{
				\begin{equation*}
					{E(Y) = \sum^{n}_{i=1} y_{i}p(y_{i})}
				\end{equation*}
				
				\begin{center}
				\begin{tabular}{|l|l|l|}
					\hline
					Y        & $p(y_{i})$       & $y_{i}.p(y_{i})$ \\ \hline
					0        & 0,6              & 0                \\ \hline
					1        & 0,25             & 0,25             \\ \hline
					2        & 0,1              & 0,2              \\ \hline
					3        & 0,05             & 0,15             \\ \hline
					\multicolumn{2}{|l|}{Total} & 0,6              \\ \hline
				\end{tabular}
				\end{center}
				
			Logo, $E(Y) = 0,6.$
			}
			
		\item Suponha que um indivíduo com $Y$ infrações reincidentes incorra em uma multa de $U\$ 100Y^{2}$. Calcule o valor esperado da multa.
		
		\solv{
			$v = 100Y^{2}\ \, \to \,\ E(v) = E(100Y^{2})\ \, \to \,\ E(v) = 100.E(Y^{2})$
			
			\begin{equation*}
				{E(Y^{2}) = \sum^{n}_{i=1} y_{i}^{2}.p(y_{i})}
			\end{equation*}
				\begin{center}
				\begin{tabular}{|l|l|l|}
					\hline
					Y        & $p(y_{i})$       & $y_{i}.p(y_{i})$ \\ \hline
					0        & 0,6              & 0                \\ \hline
					1        & 0,25             & 0,25             \\ \hline
					2        & 0,1              & 0,4              \\ \hline
					3        & 0,05             & 0,45             \\ \hline
					\multicolumn{2}{|l|}{Total} & 1,1             \\ \hline
				\end{tabular}
				\end{center}
				
		Logo, $E(Y^{2}) = 1,1$.
		
		Com isso, $E(v)=100.E(Y^{2})\ \, \to \,\ 100.1,1\ \, \to \,\ 110$.\\Valor esperado: $U\$ 110$.
		}
	\end{enumerate}

\setcounter{enumi}{4}
\item Um dado é lançado duas vezes. Seja $X$ a soma dos resultados. Calcule $E(X)$.
	
\item Um homem possui $4$ chaves em seu bolso. Como está escuro, ele não consegue ver qual a
chave correta para abrir a porta de sua casa. Ele testa cada uma das chaves até encontrar a
correta.

	\begin{enumerate}[label=\alph*)]
		\item Defina um espaço amostral para esse experimento.
		\item Defina a v.a. $X$ = número de chaves experimentadas até conseguir abrir a porta (inclusive a chave correta). 
		Quais são os valores de $X$? Qual é a função de probabilidade de $X$?
	\end{enumerate}
	
\item Seja uma v.a. $X$ com fdp dada na tabela a seguir:

    \begin{table}[htpb]
        \centering
        \begin{tabular}{|c|c|c|c|c|c|c|}
            \hline
            $X$      & 0 & 1       & 2       & 3 & 4 & 5 \\ \hline
            $P(X=x)$ & 0 & $p^{2}$ & $p^{2}$ & p & p & $p^{2}$ \\ \hline
        \end{tabular}
    \end{table}
    
    	\begin{enumerate}[label=\alph*)]
		\item Encontre o valor de $p$.
		\item Calcule $P (X \geq 4)$ e $P (X < 3)$.
		\item Calcule $P (|X − 3| \geq 2)$.
	\end{enumerate}
	\solv{}

\end{enumerate}


%\vfill
%\obs{

%O projeto em java, completo, encontra-se em: \url{https://github.com/SousaPedro11/maxmin_ed2/}

%No repositório há a documentação básica ensinando a baixar e utilizar o projeto.}
