\pagenumbering {arabic}

\section*{Variável Aleatória Discreta}
\addcontentsline{toc}{section}{Variável Aleatória Discreta}

\begin{enumerate}
\item Em um determinado condomínio residencial $30\%$ das famílias não tem filhos, $40\%$ tem $1$ filho, $20\%$ tem $2$ filhos e $10\%$ tem $3$ filhos. Seja $X$ o número de filhos de uma família sorteada ao acaso dentro desse condomínio residencial.

    \begin{enumerate}[label=\alph*)]
		\item Determine a função de probabilidade e a distribuição acumulada de $X$.
		\item Calcule a esperança e o desvio padrão de $X$.
	\end{enumerate}

	\solv{}

\item Um indivíduo que possui um seguro de automóvel de uma determinada empresa é selecionado aleatoriamente. Seja $Y$ o número de infrações ao código de trânsito para as quais o indivíduo foi reincidente nos últimos $3$ anos. A função de de probabilidade de $Y$ é:
    \begin{table}[htpb]
        \centering
        \begin{tabular}{|c|c|c|c|c|}
            \hline
            $Y$      & 0    & 1    & 2    & 3    \\ \hline
            $P(Y=y)$ & 0,60 & 0,25 & 0,10 & 0,05 \\ \hline
        \end{tabular}
    \end{table}
    
    \begin{enumerate}[label=\alph*)]
		\item Calcule $E(Y)$.
		\item Suponha que um indivíduo com $Y$ infrações reincidentes incorra em uma multa de $U\$ 100Y^{2}$. Calcule o valor esperado da multa.
	\end{enumerate}

\setcounter{enumi}{4}
\item Um dado é lançado duas vezes. Seja $X$ a soma dos resultados. Calcule $E(X)$.
	
\item Um homem possui $4$ chaves em seu bolso. Como está escuro, ele não consegue ver qual a
chave correta para abrir a porta de sua casa. Ele testa cada uma das chaves até encontrar a
correta.

	\begin{enumerate}[label=\alph*)]
		\item Defina um espaço amostral para esse experimento.
		\item Defina a v.a. $X$ = número de chaves experimentadas até conseguir abrir a porta (inclusive a chave correta). 
		Quais são os valores de $X$? Qual é a função de probabilidade de $X$?
	\end{enumerate}
	
\item Seja uma v.a. $X$ com fdp dada na tabela a seguir:

    \begin{table}[htpb]
        \centering
        \begin{tabular}{|c|c|c|c|c|c|c|}
            \hline
            $X$      & 0 & 1       & 2       & 3 & 4 & 5 \\ \hline
            $P(X=x)$ & 0 & $p^{2}$ & $p^{2}$ & p & p & $p^{2}$ \\ \hline
        \end{tabular}
    \end{table}
    
    	\begin{enumerate}[label=\alph*)]
		\item Encontre o valor de $p$.
		\item Calcule $P (X \geq 4)$ e $P (X < 3)$.
		\item Calcule $P (|X − 3| \geq 2)$.
	\end{enumerate}
	\solv{}

\end{enumerate}

\newpage
\section*{Modelos de Probabilidade Discretos}
\addcontentsline{toc}{section}{Modelos de Probabilidade Discretos}

\begin{enumerate}
\item Um atirador acerta na mosca do alvo, $20\%$ dos tiros. Qual a probabilidade de ele acertar na mosca pela primeira vez no $10$\textordmasculine\ tiro?


	\solv{}

\item Joga-se um dado equilibrado. Qual é a probabilidade de serem necessários 10 lançamentos até a primeira ocorrência de um seis?

	

\item Joga-se um dado equilibrado. Qual é a probabilidade de serem necessários 10 lançamentos até a terceira ocorrência de um seis?
	
\item Um atirador acerta na mosca do alvo, $20\%$ dos tiros. Se ele dá 10 tiros, qual a probabilidade de ele acertar na mosca no máximo 1 vez?

	
	
\item Entre os 16 programadores de uma empresa, 12 são do sexo masculino. A empresa decide sortear 5 programadores para fazer um curso avançado de programação. Qual é a probabilidade dos 5 sorteados serem do sexo masculino?

\item Uma central telefônica recebe uma média de 5 chamadas por minuto. Supondo que as chamadas que chegam constituam uma distribuição de Poisson, qual é a probabilidade de a central não receber nenhuma chamada em um minuto? e de receber no máximo 2 chamadas em 2 mintuos?

\item Seja $X$ uma v.a. aleatória binomial (n, p) com $n = 5$, $p=\dfrac{1}{3}$. Calcule $E(X^{2})$.

\item Em um certo tipo de fabricação de fita magnética, ocorrem cortes a uma taxa de um corte por 2000 pés. Qual é a probabilidade de que um rolo com comprimento de 4000 pés apresente no máximo dois cortes? Pelo menos dois cortes?

\setcounter{enumi}{9}
\item A probabilidade de uma máquina produzir uma peça defeituosa em um dia é 0,1.

	\begin{enumerate}[label=\alph*)]
		\item Qual a probabilidade de que, em 20 peças produzidas em um dia, exatamente 5 sejam defeituosas?
		\item Qual a probabilidade de que a 10ª peça produzida em um dia seja a primeira defeituosa?
	\end{enumerate}

\item Certo curso de treinamento aumenta a produtividade de uma certa população de funcionários em 80\% dos casos. Se 10 funcionários quaisquer participam deste curso, encontre a probabilidade de:
	\begin{enumerate}[label=\alph*)]
		\item exatamente 7 funcionários aumentarem a produtividade;
		\item pelo menos 3 funcionários não aumentarem a produtividade;
		\item não mais que 8 funcionários aumentarem a produtividade.
	\end{enumerate}
\end{enumerate}

\newpage
\section*{Variável Aleatória Contínua}
\addcontentsline{toc}{section}{Variável Aleatória Contínua}
\begin{enumerate}
\item Seja $X$ uma v.a. contínua cuja densidade de probabilidade é dada por:
	\begin{equation*}
		f(x)=kx^{2}\ se\ 0\leq x \leq 1.
	\end{equation*}
	
	\begin{enumerate}[label=\alph*)]
		\item Determine o valor de $k$.
		\item Calcule $P(\dfrac{1}{4}< X < \dfrac{1}{2})$.
		\item Calcule $E(X)$ e $Var(X)$.
	\end{enumerate}
	
\item O tempo de vida útil, em anos, de um eletrodoméstico é uma variável aleatória com densidade dada por
	\begin{equation*}
		f(x)=\dfrac{xe^{\frac{-x}{2}}}{4},\ x>0.
	\end{equation*}
	
	\begin{enumerate}[label=\alph*)]
		\item Mostre que $f(x)$ integra 1.
		\item Se o fabricante dá um tempo de garantia de seis meses para o produto, qual a proporção de aparelhos que devem usar essa garantia?
	\end{enumerate}
\setcounter{enumi}{3}
\item A percentagem de álcool $(100X)$ em certo composto pode ser considerada uma variável aleatória com a seguinte fdp:
	\begin{equation*}
		f(x) = 20x^{3}(1-x),\ 0 < x < 1.
	\end{equation*}

	\begin{enumerate}[label=\alph*)]
		\item Estabeleça a $FD$ de $X$.
		\item Calcule $P(X < \dfrac{2}{3})$.
		\item Suponha que o preço de venda desse composto dependa do conteúdo de álcool. Especificamente, se $\dfrac{1}{3} < X < \dfrac{2}{3}$, o composto se vende por $C_{1}$ dólares/galão, caso contrário ele se vende por $C_{2}$ dólares/galão. Se o custo $C_{3}$ dólares/galão, calcule a distribuição de probabilidade do lucro líquido por galão.
	\end{enumerate}
\end{enumerate}

\newpage
\section*{Modelos de Probabilidade Contínuos}
\addcontentsline{toc}{section}{Modelos de Probabilidade Contínuos}
\begin{enumerate}
\item Dada a v.a. $X$, uniforme em $[5, 10]$, calcule as probabilidades abaixo:
	\begin{enumerate}[label=\alph*)]
		\item $P(X < 7)$.
		\item $P(8 < X < 9)$.
		\item $P(X > 8,5)$.
		\item $P(|X-7,5| > 2)$.
	\end{enumerate}

\setcounter{enumi}{2}
\item Suponha que a duração de uma componente eletrônica possui distribuição exponencial com parâmetro $\lambda = 1$, calcule:
	\begin{enumerate}[label=\alph*)]
		\item A probabilidade de que a duração seja menor a 10.
		\item A probabilidade de que a duração esteja entre 5 e 15.
		\item O valor $t$ tal que a probabilidade de que a duração seja maior a $t$ assuma o valor 0.01.
	\end{enumerate}

\item As alturas de 10:000 alunos de um colégio têm distribuição aproximadamente normal, com média $170cm$ e desvio padrão $5cm$. Qual o número esperado de alunos com altura superior a $165cm$?

\setcounter{enumi}{10}
\item O saldo médio dos clientes de um banco é uma v.a. normal com média $R\$ 2.000,00$ e desvio padrão $R\$ 250,00$. Os clientes com os 10\% maiores saldos médios recebem tratamento VIP, enquanto aqueles com os 5\% menores saldos médios receberão propaganda extra para estimular maior movimentação da conta.
	\begin{enumerate}[label=\alph*)]
		\item Quanto você precisa de saldo médio para se tornar um cliente VIP?
		\item Abaixo de qual saldo médio o cliente receberá a propaganda extra?
	\end{enumerate}
\end{enumerate}

%\vfill
%\obs{

%O projeto em java, completo, encontra-se em: \url{https://github.com/SousaPedro11/maxmin_ed2/}

%No repositório há a documentação básica ensinando a baixar e utilizar o projeto.}
