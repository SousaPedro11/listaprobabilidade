\thispagestyle{cap2}
%\pagenumbering {arabic}
\section*{Modelos de Probabilidade Discretos}
\addcontentsline{toc}{section}{Modelos de Probabilidade Discretos}

\begin{enumerate}
\item Um atirador acerta na mosca do alvo, $20\%$ dos tiros. Qual a probabilidade de ele acertar na mosca pela primeira vez no $10$\textordmasculine\ tiro?

	\solv{
%         $
%         x = 
%         \begin{cases}
%          1 & acerta\\
%          0 & n\tilde{a}o\ acerta
%         \end{cases}
%         $
        
%         $p(0) = 0,8 = q;\ p(1) = 0,2 = p$
%         
%         \[P(Y = y) = \binom{N}{k} \cdot p^kq^{N-k}\]
        
%         $y = N = n$\textordmasculine\ de tentativas $\, \to\, y = 10$.\\
%         $k= n$\textordmasculine\ de acertos $\, \to\, k = 1$.
        $p = 0,2;\ x = 10$
        
        $P(X = 10) = 0,2^{1}\cdot 0,8^{10-1} \, \to\, (0,2) \cdot (0,8)^{9} \, \to\, (0,2) \cdot (0,8)^{9} \,\to\, 0,0268$
	}

\item Joga-se um dado equilibrado. Qual é a probabilidade de serem necessários 10 lançamentos até a primeira ocorrência de um seis?

    \solv{
        $p = \dfrac{1}{6};\ x = 10$
        
        $P(X = 10) = \dfrac{1}{6}\cdot\left(\dfrac{5}{6}\right)^{9} \, \to \, 0,0323$
    }

\item Joga-se um dado equilibrado. Qual é a probabilidade de serem necessários 10 lançamentos até a terceira ocorrência de um seis?
	
	\solv{
        $p = \dfrac{1}{6};\ x = 10;\ k = 3$
        
        $\displaystyle \binom{10 - 1}{3 - 1}\cdot\left(\dfrac{1}{6}\right)^3\cdot\left(\dfrac{5}{6}\right)^{7} \, \to\, \dfrac{9!}{2!\cdot7!}\cdot\left(\dfrac{1}{6}\right)^3\cdot\left(\dfrac{5}{6}\right)^{7} \, \to\, \dfrac{\bcancel{9}\cdot\bcancel{8}}{\bcancel{2}\cdot6\cdot\bcancel{36}}\cdot\left(\dfrac{5}{6}\right)^{7} \, \to\, \dfrac{1}{6}\cdot\left(\dfrac{5}{6}\right)^7 \, \to\, 0,0465$
	}
\item Um atirador acerta na mosca do alvo, $20\%$ dos tiros. Se ele dá 10 tiros, qual a probabilidade de ele acertar na mosca no máximo 1 vez?

	\solv{
        $p = 0,2$
        
        $P(X\leq1) = P(0) + P(1)$
        
        $P(0) = \displaystyle \binom{10}{0}\cdot(0,2)^{0}\cdot(0,8)^{10} \, \to\, \cancelto{1}{\dfrac{10!}{0!\cdot10!}}\cdot\cancelto{1}{(0,2)^{0}}\cdot(0,8)^{10} \, \to\, 0,1074$
        
        $P(1) = \displaystyle \binom{9}{1}\cdot(0,2)^{1}\cdot(0,8)^{9} \, \to\, \dfrac{10!}{1!\cdot9!}\cdot(0,2)^{1}\cdot(0,8)^{9} \, \to\, 10\cdot(0,2)\cdot(0,8)^{9} \, \to\, 0,2684$
        
        $P(X\leq1) = 0,1074 + 0,2684 \, \to\, 0,3758$
    }
	
\item Entre os 16 programadores de uma empresa, 12 são do sexo masculino. A empresa decide sortear 5 programadores para fazer um curso avançado de programação. Qual é a probabilidade dos 5 sorteados serem do sexo masculino?

    \solv{
        $P(X = 5) = \dfrac{\displaystyle\binom{12}{5}}{\displaystyle\binom{16}{5}}$
    }

\item Uma central telefônica recebe uma média de 5 chamadas por minuto. Supondo que as chamadas que chegam constituam uma distribuição de Poisson, qual é a probabilidade de a central não receber nenhuma chamada em um minuto? e de receber no máximo 2 chamadas em 2 mintuos?

\item Seja $X$ uma v.a. aleatória binomial (n, p) com $n = 5$, $p=\dfrac{1}{3}$. Calcule $E(X^{2})$.

\item Em um certo tipo de fabricação de fita magnética, ocorrem cortes a uma taxa de um corte por 2000 pés. Qual é a probabilidade de que um rolo com comprimento de 4000 pés apresente no máximo dois cortes? Pelo menos dois cortes?

\setcounter{enumi}{9}
\item A probabilidade de uma máquina produzir uma peça defeituosa em um dia é 0,1.

	\begin{enumerate}[label=\alph*)]
		\item Qual a probabilidade de que, em 20 peças produzidas em um dia, exatamente 5 sejam defeituosas?
		\item Qual a probabilidade de que a 10ª peça produzida em um dia seja a primeira defeituosa?
	\end{enumerate}

\item Certo curso de treinamento aumenta a produtividade de uma certa população de funcionários em 80\% dos casos. Se 10 funcionários quaisquer participam deste curso, encontre a probabilidade de:
	\begin{enumerate}[label=\alph*)]
		\item exatamente 7 funcionários aumentarem a produtividade;
		\item pelo menos 3 funcionários não aumentarem a produtividade;
		\item não mais que 8 funcionários aumentarem a produtividade.
	\end{enumerate}
\end{enumerate}
