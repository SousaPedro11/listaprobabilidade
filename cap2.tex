\pagestyle{cap2}
%\pagenumbering {arabic}
\section*{Modelos de Probabilidade Discretos}
\addcontentsline{toc}{section}{Modelos de Probabilidade Discretos}

\begin{enumerate}
\item Um atirador acerta na mosca do alvo, $20\%$ dos tiros. Qual a probabilidade de ele acertar na mosca pela primeira vez no $10$\textordmasculine\ tiro?

	\solv{
%         $
%         x = 
%         \begin{cases}
%          1 & acerta\\
%          0 & n\tilde{a}o\ acerta
%         \end{cases}
%         $
        
%         $p(0) = 0,8 = q;\ p(1) = 0,2 = p$
%         
%         \[P(Y = y) = \binom{N}{k} \cdot p^kq^{N-k}\]
        
%         $y = N = n$\textordmasculine\ de tentativas $\, \to\, y = 10$.\\
%         $k= n$\textordmasculine\ de acertos $\, \to\, k = 1$.
        $p = 0,2;\ x = 10$
        
        $P(X = 10) = 0,2^{1}\cdot 0,8^{10-1} \, \to\, (0,2) \cdot (0,8)^{9} \, \to\, (0,2) \cdot (0,8)^{9} \,\to\, 0,0268$
	}

\item Joga-se um dado equilibrado. Qual é a probabilidade de serem necessários 10 lançamentos até a primeira ocorrência de um seis?

    \solv{
        $p = \dfrac{1}{6};\ x = 10$
        
        $P(X = 10) = \dfrac{1}{6}\cdot\left(\dfrac{5}{6}\right)^{9} \, \to \, 0,0323$
    }

\item Joga-se um dado equilibrado. Qual é a probabilidade de serem necessários 10 lançamentos até a terceira ocorrência de um seis?
	
	\solv{
        $p = \dfrac{1}{6};\ x = 10;\ k = 3$
        
        $P(X = 10) = \displaystyle \binom{10 - 1}{3 - 1}\cdot\left(\dfrac{1}{6}\right)^3\cdot\left(\dfrac{5}{6}\right)^{7} \, \to\, \dfrac{9!}{2!\cdot7!}\cdot\left(\dfrac{1}{6}\right)^3\cdot\left(\dfrac{5}{6}\right)^{7} \, \to\, \dfrac{\bcancel{9}\cdot\bcancel{8}}{\bcancel{2}\cdot6\cdot\bcancel{36}}\cdot\left(\dfrac{5}{6}\right)^{7} \, \to\, \dfrac{1}{6}\cdot\left(\dfrac{5}{6}\right)^7$
        
        $P(X=10) = \, \to\, 0,0465$
	}
\item Um atirador acerta na mosca do alvo, $20\%$ dos tiros. Se ele dá 10 tiros, qual a probabilidade de ele acertar na mosca no máximo 1 vez?

	\solv{
        $p = 0,2$
        
        $P(X\leq1) = P(0) + P(1)$
        
        $P(0) = \displaystyle \binom{10}{0}\cdot(0,2)^{0}\cdot(0,8)^{10} \, \to\, \cancelto{1}{\dfrac{10!}{0!\cdot10!}}\cdot\cancelto{1}{(0,2)^{0}}\cdot(0,8)^{10} \, \to\, 0,1074$
        
        $P(1) = \displaystyle \binom{9}{1}\cdot(0,2)^{1}\cdot(0,8)^{9} \, \to\, \dfrac{10!}{1!\cdot9!}\cdot(0,2)^{1}\cdot(0,8)^{9} \, \to\, 10\cdot(0,2)\cdot(0,8)^{9} \, \to\, 0,2684$
        
        $P(X\leq1) = 0,1074 + 0,2684 \, \to\, 0,3758$
    }
	
\item Entre os 16 programadores de uma empresa, 12 são do sexo masculino. A empresa decide sortear 5 programadores para fazer um curso avançado de programação. Qual é a probabilidade dos 5 sorteados serem do sexo masculino?

    \solv{
        $P(X = 5) = \dfrac{\displaystyle\binom{12}{5}}{\displaystyle\binom{16}{5}} \, \to\, \dfrac{\dfrac{12!}{\bcancel{5!}\cdot7!}}{\dfrac{16!}{\bcancel{5!}\cdot11!}} \, \to\, \dfrac{12\cdot11\cdot10\cdot9\cdot8}{16\cdot15\cdot14\cdot13\cdot12} \, \to\, \dfrac{11}{14\cdot13}\cdot\dfrac{\bcancel{12}\cdot\bcancel{10}\cdot\cancelto{3}{9}\cdot\bcancel{8}}{\bcancel{16}\cdot\bcancel{15}\cdot\bcancel{12}} \, \to\, \dfrac{11\cdot3}{14\cdot13} \, \to\, \dfrac{33}{182} \, \to\, 0,1813$
    }

\item Uma central telefônica recebe uma média de 5 chamadas por minuto. Supondo que as chamadas que chegam constituam uma distribuição de Poisson, qual é a probabilidade de a central não receber nenhuma chamada em um minuto? e de receber no máximo 2 chamadas em 2 mintuos?

    \solv{
        \[P(X=x) = \dfrac{\lambda^{x}\cdot e^{-\lambda}}{x!}\]
        
        $\lambda = 5$
        
        $P(0) = \dfrac{5^{0}\cdot e^{-5}}{0!} \, \to\, \dfrac{1}{e^{5}} \, \to\, 0,0067$
        
        $P(X\leq2) = P(0) + P(1) + P(0) \, \to\, \dfrac{10^{2}}{2!\cdot e^{10}} + \dfrac{10}{e^{10}} + \dfrac{1}{e^{10}} \, \to\, \dfrac{122}{2\cdot e^{10}} \, \to\, 0,00277$
    }

\item Seja $X$ uma v.a. aleatória binomial (n, p) com $n = 5$, $p=\dfrac{1}{3}$. Calcule $E(X^{2})$.

    \solv{
        \[X \sim BN(n,p);\ E(x) = n\cdot p;\ VAR(X) = n\cdot p\cdot(1 - p)\]
        
        $E(X) = 5\cdot\dfrac{1}{3} \, \to\, \dfrac{5}{3}\\VAR(X) = \dfrac{5}{3}\cdot\left(1 - \dfrac{1}{3}\right) \, \to\, \dfrac{10}{9}$
        
        \[VAR(X) = E(x^{2}) - [E(X)]^{2} \, \to\, E(x^{2}) = VAR(X) + [E(X)]^{2}\]
        
        $E(x^2) = \dfrac{10}{9} + \left(\dfrac{5}{3}\right)^{2} \, \to\, \dfrac{35}{9}$
    }

\item Em um certo tipo de fabricação de fita magnética, ocorrem cortes a uma taxa de um corte por 2000 pés. Qual é a probabilidade de que um rolo com comprimento de 4000 pés apresente no máximo dois cortes? Pelo menos dois cortes?

    \solv{
        \[P(X=x) = \dfrac{\lambda^{x}\cdot e^{-\lambda}}{x!}\]
        
        $\lambda = 2\ \ p\diagup4000\ p\acute{e}s$
        
        $P(X\leq2) = P(2) + P(1) + P(0) \, \to\, \dfrac{2^{2}\cdot e^{-2}}{2!} + \dfrac{2\cdot e^{-2}}{1!} + \dfrac{e^{-2}}{0!} \, \to\, \dfrac{2}{e^{2}} + \dfrac{2}{e^{2}} + \dfrac{1}{e^{2}} \, \to\, 0,676676$
        
        $P(X\geq2) = 1 - P(X<2) \, \to\, 1 - [P(0) + P(1)] \, \to\, 1 - \left(\dfrac{2}{e^{2}} + \dfrac{1}{e^{2}}\right) \, \to\, 0,593994$
    }

\setcounter{enumi}{9}
\item A probabilidade de uma máquina produzir uma peça defeituosa em um dia é 0,1.

	\begin{enumerate}[label=\alph*)]
		\item Qual a probabilidade de que, em 20 peças produzidas em um dia, exatamente 5 sejam defeituosas?
		
		\solv{
            $P(X=5) = \displaystyle\binom{20}{5}\cdot(0,1)^{5}\cdot(0,9)^{15} \, \to\, \dfrac{20!}{5!\cdot15!}\cdot(0,1)^{5}\cdot(0,9)^{15} \, \to\, \dfrac{\bcancel{20}\cdot19\cdot\cancelto{3}{18}\cdot17\cdot16\cdot\bcancel{15!}}{\bcancel{5\cdot4}\cdot\bcancel{3\cdot2}\cdot1\cdot\bcancel{15!}}\cdot(0,1)^{5}\cdot(0,9)^{15} \, \to\, 19\cdot17\cdot16\cdot3\cdot(0,1)^{5}\cdot(0,9)^{15} \, \to\, 0,032$
		}
		
		\item Qual a probabilidade de que a 10ª peça produzida em um dia seja a primeira defeituosa?
		
		\solv{
            $P(X=10) = (0,1)\cdot(0,9)^9 \, \to\, 0,0387$
        }
	\end{enumerate}

\item Certo curso de treinamento aumenta a produtividade de uma certa população de funcionários em 80\% dos casos. Se 10 funcionários quaisquer participam deste curso, encontre a probabilidade de:
	\begin{enumerate}[label=\alph*)]
		\item exatamente 7 funcionários aumentarem a produtividade;
		
		\solv{
            $P(X=7) = \displaystyle\binom{10}{7}\cdot(0,8)^{7}\cdot(0,2)^{3} \, \to\, \dfrac{10\cdot\cancelto{3}{9}\cdot\cancelto{4}{8}\cdot\bcancel{7!}}{\bcancel{7!}\cdot\bcancel{3}\cdot\bcancel{2}\cdot1}\cdot(0,8)^{7}\cdot(0,2)^{3} \, \to\, 120\cdot(0,8)^{7}\cdot(0,2)^{3} \, \to\, 0,201326592$
        }
		\item pelo menos 3 funcionários não aumentarem a produtividade;
		
        \solv{
            No máximo 7 aumentaram.
            
            $P(X\leq7) = P(X\leq8) - P(X=8) \, \to\, P(X\leq8) - \displaystyle\binom{10}{8}\cdot(0,8)^{8}\cdot(0,2)^{2} \, \to\, P(X\leq8) - \dfrac{\cancelto{5}{10}\cdot9\cdot\bcancel{8!}}{\bcancel{8!}\cdot\bcancel{2!}}\cdot(0,8)^{8}\cdot(0,2)^{2} \, \to\, P(X\leq8) - 45\cdot(0,8)^{8}\cdot(0,2)^{2} \, \to\, 0,6242 - 45\cdot(0,8)^{8}\cdot(0,2)^{2} \, \to\, 0,3222$
        }
        
        \obs{$P(X\leq8)$ é calculado na letra c.}
        
		\item não mais que 8 funcionários aumentarem a produtividade.
		
		\solv{
            $P(X\leq8) = 1 - [P(X=9) + P(X=10)] \, \to\, 1 - \left[\displaystyle\binom{10}{9}\cdot(0,8)^{9}\cdot0,2 + \displaystyle\binom{10}{10}\cdot(0,8)^{10}\right]\\ \, \to\, 1-\left[\dfrac{10\cdot\bcancel{9!}}{\bcancel{9!}\cdot1!}\cdot(0,8)^{9}\cdot0,2 + (0,8)^{10}\right] \, \to\, 1 - \left[10\cdot(0,8)^{9}\cdot0,2 + (0,8)^{10}\right] \, \to\, 0,6242$
		}
	\end{enumerate}
\end{enumerate}
