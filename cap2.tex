\thispagestyle{cap2}
%\pagenumbering {arabic}
\section*{Modelos de Probabilidade Discretos}
\addcontentsline{toc}{section}{Modelos de Probabilidade Discretos}

\begin{enumerate}
\item Um atirador acerta na mosca do alvo, $20\%$ dos tiros. Qual a probabilidade de ele acertar na mosca pela primeira vez no $10$\textordmasculine\ tiro?


	\solv{}

\item Joga-se um dado equilibrado. Qual é a probabilidade de serem necessários 10 lançamentos até a primeira ocorrência de um seis?

	

\item Joga-se um dado equilibrado. Qual é a probabilidade de serem necessários 10 lançamentos até a terceira ocorrência de um seis?
	
\item Um atirador acerta na mosca do alvo, $20\%$ dos tiros. Se ele dá 10 tiros, qual a probabilidade de ele acertar na mosca no máximo 1 vez?

	
	
\item Entre os 16 programadores de uma empresa, 12 são do sexo masculino. A empresa decide sortear 5 programadores para fazer um curso avançado de programação. Qual é a probabilidade dos 5 sorteados serem do sexo masculino?

\item Uma central telefônica recebe uma média de 5 chamadas por minuto. Supondo que as chamadas que chegam constituam uma distribuição de Poisson, qual é a probabilidade de a central não receber nenhuma chamada em um minuto? e de receber no máximo 2 chamadas em 2 mintuos?

\item Seja $X$ uma v.a. aleatória binomial (n, p) com $n = 5$, $p=\dfrac{1}{3}$. Calcule $E(X^{2})$.

\item Em um certo tipo de fabricação de fita magnética, ocorrem cortes a uma taxa de um corte por 2000 pés. Qual é a probabilidade de que um rolo com comprimento de 4000 pés apresente no máximo dois cortes? Pelo menos dois cortes?

\setcounter{enumi}{9}
\item A probabilidade de uma máquina produzir uma peça defeituosa em um dia é 0,1.

	\begin{enumerate}[label=\alph*)]
		\item Qual a probabilidade de que, em 20 peças produzidas em um dia, exatamente 5 sejam defeituosas?
		\item Qual a probabilidade de que a 10ª peça produzida em um dia seja a primeira defeituosa?
	\end{enumerate}

\item Certo curso de treinamento aumenta a produtividade de uma certa população de funcionários em 80\% dos casos. Se 10 funcionários quaisquer participam deste curso, encontre a probabilidade de:
	\begin{enumerate}[label=\alph*)]
		\item exatamente 7 funcionários aumentarem a produtividade;
		\item pelo menos 3 funcionários não aumentarem a produtividade;
		\item não mais que 8 funcionários aumentarem a produtividade.
	\end{enumerate}
\end{enumerate}