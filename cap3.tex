\pagestyle{cap3}
\section*{Variável Aleatória Contínua}
\addcontentsline{toc}{section}{Variável Aleatória Contínua}
\begin{enumerate}
\item Seja $X$ uma v.a. contínua cuja densidade de probabilidade é dada por:
	\begin{equation*}
		f(x)=kx^{2}\ se\ 0\leq x \leq 1.
	\end{equation*}
	
	\begin{enumerate}[label=\alph*)]
		\item Determine o valor de $k$.
		
		\solv{
            $\displaystyle\int_{0}^{1}k\cdot x^{2}\ dx = 1 \, \to\, \left.k\cdot\dfrac{x^{3}}{3}\right|_{0}^{1} = 1 \, \to\, \dfrac{k}{3} = 1 \, \to\, k = 3$
		}
		
		\item Calcule $P\left(\frac{1}{4}< X < \frac{1}{2}\right)$.
		
        \solv{
            $P\left(\frac{1}{4}< X < \frac{1}{2}\right) = \displaystyle\int_{\sfrac{1}{4}}^{\sfrac{1}{2}}3\cdot x^{2}\ dx \, \to\, 3\cdot\displaystyle\int_{\sfrac{1}{4}}^{\sfrac{1}{2}}x^{2}\ dx \, \to\, \left. \bcancel{3}\cdot\dfrac{x^{3}}{\bcancel{3}}\right|_{\sfrac{1}{4}}^{\sfrac{1}{2}} \, \to\, \left(\dfrac{1}{2}\right)^{3} - \left(\dfrac{1}{4}\right)^{3} \, \to\, \dfrac{1}{8} - \dfrac{1}{64} \, \to\, \dfrac{7}{64}\\ \, \to\, 0,109375$
        }
        
		\item Calcule $E(X)$ e $Var(X)$.
		
		\solv{
            \[E(X) = \int_{a}^{b}x\cdot f(x)\ dx;\ E(X^{2}) = \int_{a}^{b}x^{2}\cdot f(x)\ dx;\ VAR(X) = E(X^{2}) - \left[E(X)\right]^{2}\]
            
            $E(X) = \displaystyle\int_{0}^{1}x\cdot3\cdot x^{2}\ dx \, \to\, \displaystyle\int_{0}^{1}3\cdot x^{3}\ dx \, \to\, \left.3\cdot\dfrac{x^{4}}{4}\right|_{0}^{1} \, \to\, \dfrac{3}{4} \, \to\, 0,75$
            
            $E(X^{2}) = \displaystyle\int_{0}^{1}3\cdot x^{4}\ dx \, \to\, \left.3\cdot\dfrac{x^{5}}{5}\right|_{0}^{1} \, \to\, \dfrac{3}{5} \, \to\, 0,6$
            
            $VAR(X) = 0,6 - (0,75)^{2} \, \to\, 0,0375$
		}
		
	\end{enumerate}
	
\item O tempo de vida útil, em anos, de um eletrodoméstico é uma variável aleatória com densidade dada por
	\begin{equation*}
		f(x)=\dfrac{x\cdot e^{\sfrac{-x}{2}}}{4},\ x>0.
	\end{equation*}
	
	\begin{enumerate}[label=\alph*)]
		\item Mostre que $f(x)$ integra 1.
		
		\solv{
            $F(X)=\displaystyle\int_{0}^{\infty}\dfrac{x\cdot e^{\sfrac{-x}{2}}}{4}\ dx = 1\\ \phantom{F(X)=}\dfrac{1}{4}\cdot\displaystyle\int_{0}^{\infty}x\cdot e^{\sfrac{-x}{2}}\ dx = 1 \, \to\, \left.-\dfrac{1}{2}\cdot e^{\sfrac{-x}{2}}(x+2)\right|_{0}^{\infty} = 1 \, \to\, 0 - (-1) = 1 \, \to\, 1 = 1$
		}
		\item Se o fabricante dá um tempo de garantia de seis meses para o produto, qual a proporção de aparelhos que devem usar essa garantia?
		
		\solv{
            Como a vida útil é medida em anos e a garantia é de 6 meses, a garantia é de $\dfrac{1}{2}$ ano.
            
            $P(0<X<\sfrac{1}{2}) = \displaystyle\int_{0}^{\sfrac{1}{2}}\dfrac{x\cdot e^{\sfrac{-x}{2}}}{4}\ dx \, \to\, 1 - \dfrac{5}{4\cdot e^{\sfrac{1}{4}}} \, \to\, 0,0265$
		}
		
	\end{enumerate}
	
\setcounter{enumi}{3}
\item A percentagem de álcool $(100X)$ em certo composto pode ser considerada uma variável aleatória com a seguinte fdp:
	\begin{equation*}
		f(x) = 20x^{3}(1-x),\ 0 < x < 1.
	\end{equation*}

	\begin{enumerate}[label=\alph*)]
		\item Estabeleça a $FD$ de $X$.
		
		\solv{
            $F(X) = P(X\leq x) = \displaystyle\int_{-\infty}^{x}f(x)\ dx$
            
            $F(X) = \displaystyle\int_{0}^{x}20\cdot x^{3}\cdot(1-x)\ dx \, \to\, 20\cdot\displaystyle\int_{0}^{x}\left(x^{3}-x^{4}\right)\ dx \, \to\, \left.20\cdot\left(\dfrac{x^{4}}{4} - \dfrac{x^{5}}{5}\right)\right|_{0}^{x} \, \to\, \bcancel{20}\cdot\dfrac{5\cdot x^{4} - 4\cdot x^{5}}{\bcancel{20}}\\ \, \to\, 5\cdot x^{4} - 4\cdot x^{5}$
		}
		\item Calcule $P\left(X < \frac{2}{3}\right)$.
		
		\solv{
            $F\left(\frac{2}{3}\right) = 5\cdot\left(\dfrac{2}{3}\right)^{4} - 4\cdot\left(\dfrac{2}{3}\right)^{5} \, \to\, 5\cdot\dfrac{16}{81} - 4\cdot\dfrac{32}{243} \, \to\, \dfrac{15\cdot16 - 4\cdot32}{243} \, \to\, \dfrac{240 - 128}{243} \, \to\, \dfrac{112}{243} \, \to\, 0,461$
		}
		
		\item Suponha que o preço de venda desse composto dependa do conteúdo de álcool. Especificamente, se $\dfrac{1}{3} < X < \dfrac{2}{3}$, o composto se vende por $C_{1}$ dólares/galão, caso contrário ele se vende por $C_{2}$ dólares/galão. Se o custo $C_{3}$ dólares/galão, calcule a distribuição de probabilidade do lucro líquido por galão.
		
		\solv{
            Probabilidade de ser vendida por $C_{1}$:
            
            $P\left(\frac{1}{3}<X<\frac{2}{3}\right) = F\left(\frac{2}{3}\right) - F\left(\frac{1}{3}\right)$
            
            $F\left(\frac{1}{3}\right) = 5\cdot\left(\dfrac{1}{4}\right)^{4} - 4\cdot\left(\dfrac{1}{3}\right)^{5} \, \to\, 5\cdot\dfrac{1}{81} - 4\cdot\dfrac{1}{243} \, \to\, \dfrac{15 - 4}{243} \, \to\, \dfrac{11}{243}$
            
            $P\left(\frac{1}{3}<X<\frac{2}{3}\right) = \dfrac{112 - 11}{243} \, \to\, \dfrac{101}{243} \, \to\, 0,4156$
            
            Sendo assim, a probabilidade de ser vendida por $C_{2}$ é:
            
            $1 - P\left(\frac{1}{3}<X<\frac{2}{3}\right) = 1 - \dfrac{101}{243} \, \to\, \dfrac{142}{243} \, \to\, 0,5843$
            
            Considerando $L$ como a variável que representa o lucro, temos que:
            
            $
            L =
            \begin{cases}
             C_{1} - C_{3} & \dfrac{1}{3}<X<\dfrac{2}{3}\\
             C_{2} - C_{3} & cc.
            \end{cases}
            $
            
            \begin{center}
            \begin{tabular}{|c|c|c|}
             \hline
             $L$ & $C_{1} - C_{3}$ & $C_{2} - C_{3}$\\ \hline
             $P(L = \ell)$ & $0,4156$ & $0,5843$ \\ \hline
            \end{tabular}
            \end{center}
		}
	\end{enumerate}
\end{enumerate}
