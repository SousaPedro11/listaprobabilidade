\thispagestyle{cap3}
\section*{Variável Aleatória Contínua}
\addcontentsline{toc}{section}{Variável Aleatória Contínua}
\begin{enumerate}
\item Seja $X$ uma v.a. contínua cuja densidade de probabilidade é dada por:
	\begin{equation*}
		f(x)=kx^{2}\ se\ 0\leq x \leq 1.
	\end{equation*}
	
	\begin{enumerate}[label=\alph*)]
		\item Determine o valor de $k$.
		\item Calcule $P(\dfrac{1}{4}< X < \dfrac{1}{2})$.
		\item Calcule $E(X)$ e $Var(X)$.
	\end{enumerate}
	
\item O tempo de vida útil, em anos, de um eletrodoméstico é uma variável aleatória com densidade dada por
	\begin{equation*}
		f(x)=\dfrac{xe^{\frac{-x}{2}}}{4},\ x>0.
	\end{equation*}
	
	\begin{enumerate}[label=\alph*)]
		\item Mostre que $f(x)$ integra 1.
		\item Se o fabricante dá um tempo de garantia de seis meses para o produto, qual a proporção de aparelhos que devem usar essa garantia?
	\end{enumerate}
\setcounter{enumi}{3}
\item A percentagem de álcool $(100X)$ em certo composto pode ser considerada uma variável aleatória com a seguinte fdp:
	\begin{equation*}
		f(x) = 20x^{3}(1-x),\ 0 < x < 1.
	\end{equation*}

	\begin{enumerate}[label=\alph*)]
		\item Estabeleça a $FD$ de $X$.
		\item Calcule $P(X < \dfrac{2}{3})$.
		\item Suponha que o preço de venda desse composto dependa do conteúdo de álcool. Especificamente, se $\dfrac{1}{3} < X < \dfrac{2}{3}$, o composto se vende por $C_{1}$ dólares/galão, caso contrário ele se vende por $C_{2}$ dólares/galão. Se o custo $C_{3}$ dólares/galão, calcule a distribuição de probabilidade do lucro líquido por galão.
	\end{enumerate}
\end{enumerate}