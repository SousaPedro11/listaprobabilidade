\thispagestyle{cap4}
\section*{Modelos de Probabilidade Contínuos}
\addcontentsline{toc}{section}{Modelos de Probabilidade Contínuos}
\begin{enumerate}
\item Dada a v.a. $X$, uniforme em $[5, 10]$, calcule as probabilidades abaixo:
	\begin{enumerate}[label=\alph*)]
		\item $P(X < 7)$.
		\item $P(8 < X < 9)$.
		\item $P(X > 8,5)$.
		\item $P(|X-7,5| > 2)$.
	\end{enumerate}

\setcounter{enumi}{2}
\item Suponha que a duração de uma componente eletrônica possui distribuição exponencial com parâmetro $\lambda = 1$, calcule:
	\begin{enumerate}[label=\alph*)]
		\item A probabilidade de que a duração seja menor a 10.
		\item A probabilidade de que a duração esteja entre 5 e 15.
		\item O valor $t$ tal que a probabilidade de que a duração seja maior a $t$ assuma o valor 0.01.
	\end{enumerate}

\item As alturas de 10:000 alunos de um colégio têm distribuição aproximadamente normal, com média $170cm$ e desvio padrão $5cm$. Qual o número esperado de alunos com altura superior a $165cm$?

\setcounter{enumi}{10}
\item O saldo médio dos clientes de um banco é uma v.a. normal com média $R\$ 2.000,00$ e desvio padrão $R\$ 250,00$. Os clientes com os 10\% maiores saldos médios recebem tratamento VIP, enquanto aqueles com os 5\% menores saldos médios receberão propaganda extra para estimular maior movimentação da conta.
	\begin{enumerate}[label=\alph*)]
		\item Quanto você precisa de saldo médio para se tornar um cliente VIP?
		\item Abaixo de qual saldo médio o cliente receberá a propaganda extra?
	\end{enumerate}
\end{enumerate}