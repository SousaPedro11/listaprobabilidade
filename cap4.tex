\pagestyle{cap4}
\section*{Modelos de Probabilidade Contínuos}
\addcontentsline{toc}{section}{Modelos de Probabilidade Contínuos}
\begin{enumerate}
\item Dada a v.a. $X$, uniforme em $[5, 10]$, calcule as probabilidades abaixo:
	\solv{
	\[
		f(x) = \left\{
		\begin{array}{cc}
			\dfrac{1}{\beta - \alpha}, & \alpha\leq x\leq\beta\\
			0, & c.c
		\end{array}\right.
		\, \to\,
		f(x) = \left\{
		\begin{array}{cc}
			\dfrac{1}{10 - 5}, & 5\leq x\leq10\\
			0, & c.c
		\end{array}\right.
	\]
	}
	\begin{enumerate}[label=\alph*)]
		\item $P(X < 7)$.
		
		\solv{
			$P(X < 7) = \displaystyle\int_{5}^{7}\dfrac{1}{5}\ dx \, \to\, \left. \dfrac{x}{5}\right|_{5}^{7} \, \to\, \dfrac{7-5}{5} \, \to\, \dfrac{2}{5} \, \to 0,4$
		}
		\item $P(8 < X < 9)$.
		
		\solv{
			$P(8 < X < 9) = 1 - P(X\leq8) - P(X\geq9)$
			
			$P(X\geq9)=\displaystyle\int_{8}^{9}\dfrac{1}{5}\ dx \, \to\, \left. \dfrac{x}{5}\right|_{8}^{9} \, \to\, \dfrac{9-8}{5} \, \to\, \dfrac{1}{5} \, \to\, 0,2$
		}
		
		\item $P(X > 8,5)$.
		
		\solv{
			$P(X>8,5) = \displaystyle\int_{8,5}^{10}\dfrac{1}{5}\ dx \, \to\, \left. \dfrac{x}{5}\right|_{8.5}^{10} \, \to\, \dfrac{10-8,5}{5} \, \to\, \dfrac{3}{2*5} \, \to\, 0,3$
		}
		
		\item $P(|X-7,5| > 2)$.
		
		\solv{
			$|X - 7,5| > 2 = \left\{
			\begin{array}{l}
				X - 7,5 > 2 \, \to\, X > 9,5\\
				X - 7,5 < -2 \, \to\, X<5,5
			\end{array}\right.
			$
			
			$P(|X-7,5|>2) = \displaystyle\int_{5}^{5,5}\dfrac{1}{5}\ dx + \displaystyle\int_{9,5}^{10}\dfrac{1}{5}\ dx \, \to\, \dfrac{0,5}{5} + \dfrac{0,5}{5} \, \to\, 0,2$
		}
		
	\end{enumerate}

\setcounter{enumi}{2}
\item Suponha que a duração de uma componente eletrônica possui distribuição exponencial com parâmetro $\lambda = 1$, calcule:

	\solv{
		\[
			f(x) = \left\{
			\begin{array}{cc}
				\dfrac{1}{\lambda}e^{\frac{-x}{\lambda}}, & x>0;\\
				0, & c.c
			\end{array}\right.
			\hfill,\ \lambda = 1
		\]
	}
	\begin{enumerate}[label=\alph*)]
		\item A probabilidade de que a duração seja menor a 10.
		
		\solv{
			$P(x\leq10) = 1 - e^{-10} \, \to\, 0,9999\\$
		}
		
		\item A probabilidade de que a duração esteja entre 5 e 15.
		
		\solv{
			$p(5<x<15) = P(X<15) - P(X<5) = 1 - e^{-15} - 1 + e^{-5} \, \to\, e^{-5} - e^{-15}\, \to\, 0,0067$
		}
		
		\item O valor $t$ tal que a probabilidade de que a duração seja maior a $t$ assuma o valor 0.01.
		
		\solv{
			$P(X>t) = e^{-t} = 0,01 \, \to\, \bcancel{\ln}{\bcancel{e}^{-t}} = \ln\left(\dfrac{1}{100}\right) \, \to\, t = -[\cancelto{0}{\ln(1)} - \ln(100)] \, \to\, t = \ln(100) \, \to\, t = 4,605$
		}
	\end{enumerate}

\item As alturas de 10:000 alunos de um colégio têm distribuição aproximadamente normal, com média $170cm$ e desvio padrão $5cm$. Qual o número esperado de alunos com altura superior a $165cm$?

	\solv{
		A proporção de alunos com altura superior a 165cm é dada por:
		
		$P(X > 165)$, ou $P(Z > (165 − 170)/5) = P(Z >−1) = 0,8413$
		
		 Logo, o número de alunos com mais de 165cm é uma variável aleatória com distribuiçãao binomial(10000, 0.8413).
		 
		 $X \sim B(10000,\ 0,8413);\ E(X) = n.p \, \to\, 10000.0,8413 \, \to\, 8413$
	}

\setcounter{enumi}{10}
\item O saldo médio dos clientes de um banco é uma v.a. normal com média $R\$ 2.000,00$ e desvio padrão $R\$ 250,00$. Os clientes com os 10\% maiores saldos médios recebem tratamento VIP, enquanto aqueles com os 5\% menores saldos médios receberão propaganda extra para estimular maior movimentação da conta.
	\begin{enumerate}[label=\alph*)]
		\item Quanto você precisa de saldo médio para se tornar um cliente VIP?
		\item Abaixo de qual saldo médio o cliente receberá a propaganda extra?
	\end{enumerate}
\end{enumerate}