\pagestyle{cap1}
\pagenumbering {arabic}

\section*{Lista Substitutiva}
\addcontentsline{toc}{section}{Lista Substitutiva}

\begin{enumerate}
\item Uma urna contém 2 bolas vermelhas, 3 verdes e 2 azuis. Duas bolas são sorteadas aleatoriamente. Qual é a probabilidade de que nenhuma das bolas sejam azuis?

\item Existem 3 moedas. Uma moeda que possui duas caras, a outra possui probabilidade 0,75 de resultar em cara e a terceira é uma moeda justa.
Uma moeda selecionada aleatoriamente é lançada.

	\solv{
		\begin{center}
        		\begin{tabular}{|c|c|c|c|}
           		 \hline
           		 	&  Moeda Duas Caras & Moeda Viciada  & Moeda Honesta \\ \hline
           		 Cara & 1 & $\dfrac{3}{4}$ & $\dfrac{1}{2}$ \\ \hline
           		 Coroa & 0 & $\dfrac{1}{4}$ & $\dfrac{1}{2}$ \\ \hline
        		\end{tabular}
    		\end{center}
	}

	\begin{enumerate}[label=\alph*.]
    		\item Qual é a probabilidade de sair cara?
		
		\item Dado que ocorreu cara, qual é a probabilidade de a moeda selecionado ser a moeda honesta?
	\end{enumerate}

\item Um dado não honesto é lançado. A v.a. discreta $X$ representa a face voltada pra cima. A função de probabilidade de $X$ é dada abaixo:

    \begin{center}
        \begin{tabular}{|c|c|c|c|c|c|c|}
            \hline
            $X$          & 1 & 2  & 3 & 4 & 5 & 6 \\ \hline
            $P(X=x)$ & $a$ & $a$ & $a$ & $b$ & $b$ & 0,3 \\ \hline
        \end{tabular}
    \end{center}

	\solv{
		\[\sum\limits_{i=1}^np(x_{i}) = 1;\ E(X) = \sum^{n}_{i=1} x_{i}p(x_{i});\ E(X^{2}) = \sum^{n}_{i=1} x_{i}^{2}p(x_{i});\ Var(X) = E(X^{2}) - [E(X)]^{2}\]
	}
	\begin{enumerate}[label=\alph*.]
		\item Dado que $E(X) = 4,2$, encontre o valor de $a$ e $b$.
		
			\solv{
				Como $\displaystyle\sum\limits_{i=1}^np(x_{i}) = 1$, temos: $3a + 2b + 0,3 = 1 \, \to\, 3a + 2b = 0,7$
				
				Como $E(X) = \displaystyle\sum^{n}_{i=1} x_{i}p(x_{i})$, temos: $a + 2a + 3a + 4b + 5b + 1,8 = 4,2 \, \to\, 6a + 9b = 4,2 - 1,8\\ \, \to\, 6a + 9b = 2,4$
				
				Com isso podemos resolver da seguinte forma:
				
				$
				\left\{
				\begin{array}{cc}
					3a + 2b & = 0,7\\
					6a + 9b & = 2,4
				\end{array}
				\right.
				$
				$\xrightarrow{Eq_{2}^{\backprime} \rightarrow Eq_{2}-2Eq_{1}}$
				$
				\left\{
				\begin{array}{cc}
					3a + 2b & = 0,7\\
					0a + 5b & = 1
				\end{array}
				\right.
				$
				$\xrightarrow{Eq_{2}^{\backprime\backprime} \rightarrow \sfrac{Eq_{2}^{\backprime}}{5}}$
				$
				\left\{
				\begin{array}{cc}
					3a + 2b & = 0,7\\
					0a + \phantom{1}b & = 0,2
				\end{array}
				\right.
				$
				
				$\xrightarrow{Eq_{1}^{\backprime} \rightarrow Eq_{1}-2Eq_{2}^{\backprime\backprime}}$
				$
				\left\{
				\begin{array}{cc}
					3a + 0b & = 0,3\\
					0a + \phantom{1}b & = 0,2
				\end{array}
				\right.
				$
				$\xrightarrow{Eq_{1}^{\backprime\backprime} \rightarrow \sfrac{Eq_{1}^{\backprime}}{3}}$
				$
				\left\{
				\begin{array}{cc}
					\phantom{1}a + 0b & = 0,1\\
					0a + \phantom{1}b & = 0,2
				\end{array}
				\right.
				$
				
				Logo, $a=0,1$ e $b=0,2$
			}
		\item Mostre que $E(X^{2}) = 20,4$.
		
			\solv{
				Como $E(X^{2}) = \displaystyle\sum^{n}_{i=1} x_{i}^{2}p(x_{i})$, temos: $a+4a+9a+16b+25b+36\cdot0,3 \, \to\, 14a + 41b + 10,8$
				
				Substituindo $a$ e $b$ por seus valores chegamos a:\\$E(X^{2}) = 14\cdot0,1 + 41\cdot0,2 + 10,8 \, \to\, 1,4 + 8,2 + 10,8 \, \to\, 20,4$
			}
		
		\item Encontre $Var(5-3X)$.
		
		\solv{
			Propriedades da Variancia ($c$ é uma constante):
			\begin{enumerate}
				\item $Var(X+c) = Var(X)$
				\item $Var(cX) = c^{2}Var(X)$
				\item $Var(X + Y) = Var(X) + Var(Y)$
			\end{enumerate}
			
			Utilizando a propriedade i. temos: $Var(5 + 3X) = Var(3X)$;\\Utilizando a propriedade ii. no resultado obtido chegamos a: $Var(3X) = 3^{2}Var(X)$.
			
			Logo, $Var(5-3X) = 3^{2}Var(X) \, \to\, 9Var(X)$
			
			$Var(X) = E(X^{2}) - [E(X)]^{2}\, \to\, Var(X) = 20,4 - (4,2)^{2} \, \to\, Var(X) = 20,4 - 17,64 \, \to\, Var(X) = 2,76$
			
			Com isso, $Var(5-3X) = 9\cdot2,76 \, \to\, 24,84$
		}
		
	\end{enumerate}
	
\item Considere que $X$ seja uma v.a. contínua tal que sua f.d.p. é

\[f(x)=c\cdot x^{n},\ 0<x<1\]

	\begin{enumerate}[label=\alph*)]
		\item Encontre o valor de c.
		
			\solv{
				$\displaystyle\int_{0}^{1}c\cdot x^{n}\ dx = 1 \, \to\, \left.c\cdot \dfrac{x^{n+1}}{n+1}\right|_{0}^{1}=1 \, \to\, c\cdot \dfrac{\cancelto{1}{1^{n+1}}-\cancelto{0}{0^{n+1}}}{n+1} = 1 \, \to\, \dfrac{c}{n+1} = 1 \, \to\, c = n+1$
			}
		
		\item $P(X > k)$.
		
			\solv{
				$P(X>k) = \displaystyle\int_{k}^{1}(n+1)\cdot x^{n}\ dx \, \to\, \left.x^{n+1}\right|_{k}^{1} \, \to\, \cancelto{1}{1^{n+1}} - k^{n+1} \, \to\, 1-k^{n+1}$
			}
		
	\end{enumerate}
	
\item Um agricultor cultiva laranjas e também produz mudas para vender. Após alguns meses a muda pode ser atacada por fungos com probabilidade 0,02 e, nesse caso, ela tem probabilidade 0,5 de ser recuperável. O custo de cada muda produzida é R\$ 1,20, que será acrescido de mais R\$ 0,50 se precisar ser recuperada. As irrecuperáveis são descartadas.
Sabendo que cada muda é vendida a R\$ 3,50, encontre a distribuição da variável aleatória "lucro por muda produzida".

	\begin{enumerate}[label=\alph*.]
		\item Qual é o lucro médio por muda produzida?
		\item Em uma plantação de 10000 mudas, qual é o lucro esperado?
		\item Em um lote de 50 mudas, qual é a probabilidade de que pelo menos 45 sejam aproveitáveis?
	\end{enumerate}

\item O número de milhas que um determinado carro pode percorrer antes que a bateria se esgote é distribuído exponencialmente com uma média de 10.000 milhas. O proprietário do carro precisa fazer uma viagem de 5000 milhas. Qual é a probabilidade de que ele será capaz de completar a viagem sem ter que substituir a bateria do carro?

\end{enumerate}


%\vfill
%\obs{

%O projeto em java, completo, encontra-se em: \url{https://github.com/SousaPedro11/maxmin_ed2/}

%No repositório há a documentação básica ensinando a baixar e utilizar o projeto.}
