\documentclass[a4paper, 10pt, oneside]{article} %book, article ou report
\usepackage[brazil]{babel} %linguagem do documento
\usepackage[utf8]{inputenc} %reconhece acento e cedilha
\usepackage{amssymb, amsmath, pxfonts} %permite simbolos matemáticos
\usepackage{mathrsfs} %permite uso de fontes para conjuntos
\usepackage[normalem]{ulem} %permite sublinhar palavras
\usepackage{mathrsfs} %permite o uso de letras trabalhadas
\usepackage{textcomp}
\usepackage{enumitem}
\usepackage{cancel}
\usepackage{footmisc}
\usepackage{tocloft}
\renewcommand{\cftsecleader}{\color{red}{\cftdotfill{\cftdotsep}}}
% \usepackage{array}
\usepackage{xfrac}
\usepackage{fancyhdr}
%\usepackage{times}
%\renewcommand{\familydefault}{\sfdefault}
%\usepackage{subfigure}
\usepackage[top=3cm,left=3cm,right=2cm,bottom=2cm]{geometry} %margens
\usepackage{setspace}
\usepackage{graphicx} %permite inserir figuras
\usepackage{adjustbox}
\usepackage[usenames,dvipsnames]{color} %permite letras coloridas
\usepackage{makeidx} %pra criar índice remissivo
\makeindex %construção do índice
\usepackage[pdftex,bookmarksnumbered,colorlinks]{hyperref} %hiperlinks coloridos
\DeclareMathAlphabet {\mathbfit}{OML}{cmm}{b}{it} %negrito e itálico
\usepackage{indentfirst}
\usepackage{mathptmx}
\usepackage{anyfontsize}
\usepackage{t1enc}
\usepackage{longtable}
\usepackage{wrapfig,lipsum,booktabs}
\usepackage{booktabs,multirow}
\newcommand{\solv}[1] {{\color{red}{#1}}}
\newcommand{\obs}[1] {{\sl{\color{blue}{OBS: {#1}}}}}
\newcommand{\bfit}[1]{\sl{\textbf{#1}}}
\newcommand{\mai}[1]{\uppercase{#1}}% Maiúsculas
\newcommand{\mb}[1]{\mai{\textbf{#1}}}% Maiúsculas + Negrito
\newcommand{\mbi}[1]{\uppercase{\bfit{#1}}}% Maiúsculas + Negrito e Itálico
\newcommand{\mam}[3]{\multicolumn{#1}{#2}{\mai{#3}}}% Multicolunas + Maiúsculas
\newcommand{\mbm}[3]{\multicolumn{#1}{#2}{\mbi{#3}}}% Multicolunas + Negrito e Itálico

\DeclareUnicodeCharacter{2212}{-}

\fancypagestyle{cap1}{
\fancyhead[R]{\thepage}
\fancyhead[L]{\textsl{\uppercase{Variável\ Aleatória\ Discreta}}}
% \fancyhead[RE]{\textsl{\uppercase{Variável\ Aleatória\ Discreta}}}
\fancyfoot{}
}

\fancypagestyle{cap2}{
\fancyhead[R]{\thepage}
\fancyhead[L]{\textsl{\uppercase{Modelos\ de\ Probabilidade\ Discretos}}}
% \fancyhead[RE]{\textsl{\uppercase{Modelos\ de\ Probabilidade\ Discretos}}}
\fancyfoot{}
}

\fancypagestyle{cap3}{
\fancyhead[R]{\thepage}
\fancyhead[L]{\textsl{\uppercase{Variável\ Aleatória\ Contínua}}}
% \fancyhead[RE]{\textsl{\uppercase{Variável\ Aleatória\ Contínua}}}
\fancyfoot{}
}

\fancypagestyle{cap4}{
\fancyhead[R]{\thepage}
\fancyhead[L]{\textsl{\uppercase{Modelos\ de\ Probabilidade\ Contínuos}}}
% \fancyhead[RE]{\textsl{\uppercase{Modelos\ de\ Probabilidade\ Contínuos}}}
\fancyfoot{}
}
\usepackage{listings}
\usepackage{xcolor}         % Pacote de cores
\usepackage[skip=2pt,font=scriptsize]{caption}

% Define cores em RGB 
\definecolor{javared}{rgb}{0.6,0,0} % for strings
\definecolor{javagreen}{rgb}{0.25,0.5,0.35} % comments
\definecolor{javapurple}{rgb}{0.5,0,0.35} % keywords
\definecolor{javadocblue}{rgb}{0.25,0.35,0.75} % javadoc
 
% Configuração para exibir código em Java
\lstset{language=Java, % Linguagem de programação
inputencoding=latin1,
basicstyle=\ttfamily\scriptsize, % Tamanho da fonte do código
keywordstyle=\color{javapurple}\bfseries, % Estilo das palavras chaves
stringstyle=\color{javared}, % Estilo de Strings
commentstyle=\color{javagreen}, % Estilo dos Comentários
morecomment=[s][\color{javadocblue}]{/**}{*/},
numbers = left, % Posição da numeração das linhas
numberstyle = \tiny\color{black}, % Estilo da numeração de linhas
stepnumber=2, % Numeração das linhas ocorre a cada quantas linhas?
numbersep=10pt, % Distância entre a numeração das linhas e o código
tabsize=4, % Configura tabulação em x espaços
showspaces=false, % Exibe espaços com um sublinhado
showstringspaces=false, % Sublinha espaços em Strings
backgroundcolor = \color{white}, % Cor de fundo
showtabs = false, % Exibe tabulação com um sublinhado
frame = trBL, % Envolve o código com uma moldura, pode ser single ou trBL
rulecolor = \color{black}, % Cor da moldura
captionpos = b, % Posição do título pode ser t (top) ou b (bottom)
breaklines = true, % Configura quebra de linha automática
breakatwhitespace= false, % Configura quebra de linha
title = \lstname, % Exibe o nome do arquivo incluido
}

\lstset{literate=
  {á}{{\'a}}1 {é}{{\'e}}1 {í}{{\'i}}1 {ó}{{\'o}}1 {ú}{{\'u}}1
  {Á}{{\'A}}1 {É}{{\'E}}1 {Í}{{\'I}}1 {Ó}{{\'O}}1 {Ú}{{\'U}}1
  {à}{{\`a}}1 {è}{{\`e}}1 {ì}{{\`i}}1 {ò}{{\`o}}1 {ù}{{\`u}}1
  {À}{{\`A}}1 {È}{{\'E}}1 {Ì}{{\`I}}1 {Ò}{{\`O}}1 {Ù}{{\`U}}1
  {ä}{{\"a}}1 {ë}{{\"e}}1 {ï}{{\"i}}1 {ö}{{\"o}}1 {ü}{{\"u}}1
  {Ä}{{\"A}}1 {Ë}{{\"E}}1 {Ï}{{\"I}}1 {Ö}{{\"O}}1 {Ü}{{\"U}}1
  {â}{{\^a}}1 {ê}{{\^e}}1 {î}{{\^i}}1 {ô}{{\^o}}1 {û}{{\^u}}1
  {Â}{{\^A}}1 {Ê}{{\^E}}1 {Î}{{\^I}}1 {Ô}{{\^O}}1 {Û}{{\^U}}1
  {œ}{{\oe}}1 {Œ}{{\OE}}1 {æ}{{\ae}}1 {Æ}{{\AE}}1 {ß}{{\ss}}1
  {ç}{{\c c}}1 {Ç}{{\c C}}1 {ø}{{\o}}1 {å}{{\r a}}1 {Å}{{\r A}}1
  {€}{{\EUR}}1 {£}{{\pounds}}1 {ã}{{\~{a}}}1 {õ}{{\~{o}}}1 
  {Õ}{{\~{O}}}1
}
